\documentclass{scrartcl}

\usepackage[utf8]{inputenc}
\usepackage[T1]{fontenc}
\usepackage{lmodern}
\usepackage[ngerman]{babel}
\usepackage{amsmath}
\usepackage{amssymb}

\usepackage{graphicx}
\graphicspath{ {./img/} }


\title{Procedural City Generation}
\author{Jakob Krause and Karl Volkenandt}
\begin{document}

\maketitle
\tableofcontents
\newpage

\section{Overview over possible solutions}
There are multiple possible solutions to this problemwe found out about or thought
about doing.
Erst terrain generation mittels noise
Ideen/Ansätze: Agent based, probability dist based, l2 grammars
Kurzübersicht

\subsection{Probability distribution}
One thought was to pick a center for the city and ditribute streets and buildings
around it according to a normal distribution so that there is a higher density of streets
and buildings in the middle than further away from the center. This is a rather vague
idea since it does not explain how we would go about creating the street network.

\subsection{Lindenmayer system}
Lindenmayer systems or short L-systems are a way of generating complicated patterns
with only a few given rules that were initially inspired by the way plants evolve.
The basic idea is to give a set of possible symbols that are treated as either variables
or constants and a set of transformation rules describing how variables can be replaced.
constants are not allowed to be replaced. We then give an initial string (also
called an axiom) and iterate: each symbol of our initial state gets replaced via the given rules.
When done we get a new string consiting of elements of our variables and constants
and repeat the process. After iterating for some time we can stop and need to
translate the output string into an image for example by interprating every character
of the string as an instruction. Via this method many self similar
structures can be created such as the dragon curve or the Sierpinski triangle.
While this method can give some nice looking and complicated patterns we thought
it would look too artificial when applied to street patterns and did not look further
into it.


\subsection{Agent based}
Another method to generate a believable city that takes care of the street network
and the buildings simultaniously is to simulate the process of
city-generation. This is done in the agent based approach where different agents
move around and interact with the enviroment based on the local perspective.
Possible agents include street builders, house builders and people investing in certain
areas resulting in industrial quarters or housing areas. This would also add new
buildings but can make the process more complex.



\section{Our approaches}
Before we started thinking how we built our city we wanted to create a terrain
on which the city should be built. We then followed with two approaches to the
city generation and found a way to visualize the results.
From now on everything happens on a two dimensional grid which we wish to fill
with a natural looking city and the terrain around it.


\subsection{Terrain generation using simplex noise}

For terrain generation we used a simplex noise. This is a noise that is similar
to the computationally more expensive Perlin noise which is used for tasks like
procedural texture generation and is used in the terrain generation of the computer
game minecraft. This noise is procedural generated the result is not random. To
obtain mentioned randomness and get different terrains we generate
a three dimensional simplex noise and use only a random two dimensional slice as our terrain.
Using this method we get some natural looking two dimensional terrains consisting
of three levels of heigt: water, grass and mountain.
\begin{figure}
  \centering
  \includegraphics[scale = 0.2]{terrain_example}
  \caption{An example of a terrain generated by simplex noise \\Blue = Water, Green = Grass, Brown = Mountain}
\end{figure}
The next step is to figure out where we can start to build the city. This is done
by assigning a grade to each point on the grid and starting tha city at the point
with the highest score. This is done by specifying a radius (vaguely giving
the size of the city) and checking for each point how much area of grass,
water and mountain is contained in a circle of the given radius.
We decided that starting a city near mountain is not preferred and thus the number
of tiles representing mountains is subtracted from the score. Conversely the area
of grass is added to the score. For water we thought that a certain amount of water
is nice but to much or to little water is bad. To realize this we add a weighted score based
on the difference of the preferred amount of water and the real amount of water in
the circle. The prefered amount of water is given by a percentage that should be
covered in water. This percentage then influences how far away from water the
city will be built.
\begin{figure}
  \centering
  \includegraphics[scale = 0.25]{heatmap_citycenter}
  \includegraphics[scale = 0.2]{terrain_citycenter_circle}
  \caption{(left) A heatmap visualizing the score assigned to each position on the map\\
          (right) the final chosen point inside the red circle where the city will be built}
\end{figure}

\subsection{Agent based}
To pursue the idea of an agent based system we would first like to have a street
leading to the city center. This makes thematically sense since cities do not get
built in the middle of nothing. To do this we select a random point on the border
of our map and connect it with a street to the city center. To do this in a way
that does not look artificial we equally distribute points every few tiles on
the straight line between the point on the border and the point chosen to be our
city center. We then move these points by a small random amount and use them as the
control points for a bezier curve forming the street coming from the outside to
the city.

We now introduce two types of agents to our map: The street builder and the house builder.
All agents have a method telling them where to move and a method telling them
how to change the enviroment. The way how they determine where to move is very similar.
Theoretically the agents look at each direction they can move (one tile north, east, south or west)
and assign a score to them. We then look at the two largest scores and the agent
moves towards one of the two directions at random with a higher chance of chosing
the direction in which the maximum score is assigned. This is done in order to
get rid of the deterministic movement and allow for multiple agents to move along
different paths. Instead of just looking at the relevant positions on the map we
look at each and every single point on our map and evaluate a score. That way we
can plot this scoremap and estimate how the agents will move based on the given
parameters.


\subsection{Procedural street generation via voronoi}



erster schritt: finde optimalen stadtstartpunkt und baue von da aus
  wasser/gras

stadtgenerierung (Beschreibung):

\subsection{Anzeige/Plotting}
Tileplotting mittels pygame, anfangs nur Farbkästen, dann sprites




\end{document}
